\documentclass[11pt,letterpaper]{article}
\usepackage[utf8]{inputenc}
\usepackage[spanish,es-tabla]{babel}
\usepackage[margin=2.5cm]{geometry}
\usepackage{graphicx}
\usepackage{float}
\usepackage{xcolor}
\usepackage{tcolorbox}
\usepackage{hyperref}
\usepackage{enumitem}
\usepackage{amsmath}
\usepackage{amssymb}
\usepackage{tabularx}
\usepackage{booktabs}
\usepackage{multirow}
\usepackage{array}
\usepackage{longtable}
\usepackage{colortbl}
\usepackage{fancyhdr}
\usepackage{lastpage}
\usepackage{tikz}
\usetikzlibrary{shapes.geometric, arrows, positioning, calc}

% Configuración de colores institucionales
\definecolor{usachblue}{RGB}{0,46,93}
\definecolor{usachred}{RGB}{200,16,46}
\definecolor{darkgreen}{RGB}{0,100,0}
\definecolor{warningyellow}{RGB}{255,193,7}
\definecolor{infoblue}{RGB}{52,152,219}

% Configuración de cajas
\tcbuselibrary{skins,breakable}

\newtcolorbox{requirements}[1][]{
    colback=infoblue!10,
    colframe=infoblue!60,
    fonttitle=\bfseries,
    title=Requisitos Mínimos,
    breakable,
    #1
}

\newtcolorbox{deliverables}[1][]{
    colback=darkgreen!10,
    colframe=darkgreen!60,
    fonttitle=\bfseries,
    title=Entregables,
    breakable,
    #1
}

\newtcolorbox{evaluation}[1][]{
    colback=usachred!10,
    colframe=usachred!60,
    fonttitle=\bfseries,
    title=Criterios de Evaluación,
    breakable,
    #1
}

\newtcolorbox{warning}[1][]{
    colback=warningyellow!10,
    colframe=warningyellow!80,
    fonttitle=\bfseries,
    title=⚠ Importante,
    breakable,
    #1
}

% Encabezado y pie de página
\pagestyle{fancy}
\fancyhf{}
\lhead{\textcolor{usachblue}{Geoinformática - USACH}}
\chead{\textcolor{usachblue}{Primera Evaluación de Proyecto}}
\rhead{\textcolor{usachblue}{2025-1}}
\lfoot{Prof. Francisco Parra O.}
\cfoot{Página \thepage\ de \pageref*{LastPage}}
\rfoot{\today}

\title{
    \vspace{-2cm}
    \Large{\textbf{UNIVERSIDAD DE SANTIAGO DE CHILE}} \\
    \large{Facultad de Ingeniería} \\
    \large{Departamento de Ingeniería Informática} \\
    \vspace{1cm}
    \LARGE{\textbf{PRIMERA EVALUACIÓN DE PROYECTO}} \\
    \Large{\textbf{Análisis Geoespacial Aplicado}} \\
    \vspace{0.5cm}
    \large{Presentación de Avances e Informe Técnico}
    \vspace{1cm}
}

\author{
    Prof. Francisco Parra O. \\
    \texttt{francisco.parra@usach.cl}
}

\date{
    \textbf{Fecha de publicación:} \today \\
    \textbf{Fecha de presentaciones:} 7 de octubre de 2025 \\
    \textbf{Fecha de entrega del informe:} 10 de octubre de 2025
}

\begin{document}

\maketitle

\section{Descripción General}

Esta primera evaluación constituye un \textbf{hito fundamental} en el desarrollo del proyecto semestral de su grupo. Después de 4 semanas desde la selección del proyecto (2 ya transcurridas + 2 adicionales), deben demostrar avances concretos en:

\begin{itemize}[leftmargin=*]
    \item \textbf{Comprensión del problema:} Marco teórico y justificación clara
    \item \textbf{Adquisición de datos:} Fuentes identificadas y datos descargados
    \item \textbf{Análisis exploratorio:} Primeras visualizaciones y estadísticas
    \item \textbf{Pipeline técnico:} Ambiente de desarrollo funcional
    \item \textbf{Plan de trabajo:} Cronograma realista para el resto del semestre
\end{itemize}

\section{Objetivos de la Evaluación}

\subsection{Objetivos Principales}
\begin{enumerate}
    \item \textbf{Verificar factibilidad:} Confirmar que el proyecto es realizable con los datos y tiempo disponibles
    \item \textbf{Evaluar comprensión:} Asegurar que entienden el problema geoespacial a resolver
    \item \textbf{Validar enfoque técnico:} Revisar que las herramientas y métodos sean apropiados
    \item \textbf{Proporcionar retroalimentación:} Guiar ajustes necesarios antes de continuar
\end{enumerate}

\section{Componentes de la Evaluación}

\subsection{Distribución de Ponderaciones}

\begin{table}[H]
\centering
\begin{tabular}{|l|c|p{8cm}|}
\hline
\rowcolor{usachblue!20}
\textbf{Componente} & \textbf{Peso} & \textbf{Descripción} \\
\hline
Informe Técnico & 40\% & Documento escrito con todos los requisitos \\
\hline
Presentación Oral & 25\% & Exposición de 15 minutos + 5 de preguntas \\
\hline
Código y Reproducibilidad & 20\% & Repositorio GitHub con código funcional \\
\hline
Visualizaciones & 10\% & Calidad de mapas y gráficos producidos \\
\hline
Trabajo en Equipo & 5\% & Evidencia de colaboración efectiva \\
\hline
\textbf{Total} & \textbf{100\%} & \\
\hline
\end{tabular}
\caption{Ponderación de componentes de evaluación}
\end{table}

\section{Requisitos Mínimos por Componente}

\subsection{A. Informe Técnico (40\%)}

\begin{requirements}
\textbf{Extensión:} 15-20 páginas (sin incluir anexos) \\
\textbf{Formato:} LaTeX o Markdown convertido a PDF \\
\textbf{Estructura obligatoria:}
\end{requirements}

\subsubsection{Estructura del Informe}

\begin{enumerate}[label=\arabic*.]
    \item \textbf{Resumen Ejecutivo} (1 página)
    \begin{itemize}
        \item Problema abordado
        \item Área de estudio
        \item Metodología propuesta
        \item Resultados preliminares
    \end{itemize}

    \item \textbf{Introducción} (2-3 páginas)
    \begin{itemize}
        \item Contexto del problema
        \item Justificación de la relevancia
        \item Objetivos específicos
        \item Preguntas de investigación
    \end{itemize}

    \item \textbf{Marco Teórico} (3-4 páginas)
    \begin{itemize}
        \item Revisión de literatura (mínimo 10 referencias)
        \item Conceptos geoespaciales clave
        \item Casos de estudio similares
        \item Estado del arte en la temática
    \end{itemize}

    \item \textbf{Área de Estudio} (2-3 páginas)
    \begin{itemize}
        \item Delimitación geográfica precisa
        \item Características territoriales relevantes
        \item Mapa de ubicación (obligatorio)
        \item Justificación de la selección
    \end{itemize}

    \item \textbf{Datos y Metodología} (4-5 páginas)
    \begin{itemize}
        \item \textbf{Fuentes de datos:}
        \begin{itemize}
            \item Tabla con todas las fuentes
            \item Descripción de cada dataset
            \item Fechas de actualización
            \item Resolución espacial/temporal
        \end{itemize}
        \item \textbf{Procesamiento preliminar:}
        \begin{itemize}
            \item Diagrama de flujo metodológico
            \item Herramientas utilizadas
            \item Transformaciones aplicadas
        \end{itemize}
    \end{itemize}

    \item \textbf{Resultados Preliminares} (3-4 páginas)
    \begin{itemize}
        \item Análisis exploratorio de datos (EDA)
        \item Estadísticas descriptivas espaciales
        \item Primeras visualizaciones
        \item Patrones identificados
    \end{itemize}

    \item \textbf{Cronograma} (1 página)
    \begin{itemize}
        \item Gantt chart con actividades restantes
        \item Hitos principales
        \item Distribución de responsabilidades
    \end{itemize}

    \item \textbf{Conclusiones Preliminares} (1 página)
    \begin{itemize}
        \item Factibilidad del proyecto
        \item Desafíos identificados
        \item Ajustes propuestos
    \end{itemize}
\end{enumerate}

\subsection{B. Presentación Oral (25\%)}

\begin{deliverables}
\textbf{Duración:} 15 minutos de exposición + 5 minutos de preguntas \\
\textbf{Formato:} Presencial con apoyo de slides \\
\textbf{Participación:} Todos los integrantes del grupo deben participar en la exposición
\end{deliverables}

\subsubsection{Estructura de la Presentación}

\begin{enumerate}
    \item \textbf{Introducción} (2 min)
    \begin{itemize}
        \item Problema y motivación
        \item Objetivos del proyecto
    \end{itemize}

    \item \textbf{Área de Estudio} (2 min)
    \begin{itemize}
        \item Mapas de localización
        \item Características relevantes
    \end{itemize}

    \item \textbf{Datos y Métodos} (4 min)
    \begin{itemize}
        \item Fuentes de datos (con ejemplos visuales)
        \item Pipeline de procesamiento
        \item Herramientas utilizadas
    \end{itemize}

    \item \textbf{Resultados Preliminares} (5 min)
    \begin{itemize}
        \item Demo en vivo (opcional pero valorado)
        \item Visualizaciones principales
        \item Hallazgos iniciales
    \end{itemize}

    \item \textbf{Próximos Pasos} (2 min)
    \begin{itemize}
        \item Plan de trabajo
        \item Desafíos esperados
    \end{itemize}
\end{enumerate}

\subsection{C. Código y Reproducibilidad (20\%)}

\begin{requirements}
\textbf{Repositorio GitHub:} \texttt{github.com/[usuario]/geoinformatica-proyecto} \\
\textbf{Estructura mínima requerida:}
\end{requirements}

\begin{verbatim}
proyecto/
├── README.md                 # Documentación principal
├── requirements.txt          # Dependencias Python
├── data/
│   ├── raw/                 # Datos originales (o links de descarga)
│   └── processed/           # Datos procesados
├── notebooks/
│   ├── 01_exploratory.ipynb # Análisis exploratorio
│   ├── 02_preprocessing.ipynb
│   └── 03_analysis.ipynb
├── src/
│   ├── data_download.py    # Scripts de descarga
│   ├── preprocessing.py    # Funciones de procesamiento
│   └── visualization.py    # Funciones de visualización
├── outputs/
│   ├── figures/            # Gráficos generados
│   └── maps/              # Mapas producidos
└── docs/
    └── informe_v1.pdf     # Informe de primera entrega
\end{verbatim}

\subsubsection{Requisitos de Código}

\begin{itemize}
    \item \textbf{Documentación:} Cada función debe tener docstring
    \item \textbf{Reproducibilidad:} Instrucciones claras para ejecutar
    \item \textbf{Datos:} Si son >100MB, incluir script de descarga
    \item \textbf{Ambiente:} Archivo \texttt{requirements.txt} o \texttt{environment.yml}
\end{itemize}

\subsection{D. Visualizaciones (10\%)}

\begin{evaluation}
Se evaluará la calidad técnica y comunicativa de las visualizaciones
\end{evaluation}

\subsubsection{Requisitos Mínimos de Visualización}

\begin{itemize}
    \item \textbf{3 mapas temáticos} como mínimo:
    \begin{itemize}
        \item Mapa de ubicación del área de estudio
        \item Mapa de datos principales
        \item Mapa de análisis o resultado preliminar
    \end{itemize}
    \item \textbf{5 gráficos estadísticos}:
    \begin{itemize}
        \item Histogramas de variables clave
        \item Series temporales (si aplica)
        \item Correlaciones espaciales
        \item Diagramas de dispersión
    \end{itemize}
    \item \textbf{1 visualización interactiva} (Folium, Plotly, o Streamlit)
\end{itemize}

\section{Checklist de Entregables}

\begin{warning}
Todos los items de esta lista son \textbf{obligatorios}. La ausencia de cualquiera resultará en descuento de puntaje.
\end{warning}

\subsection{Checklist Completo}

\begin{longtable}{|p{1cm}|p{10cm}|c|}
\hline
\rowcolor{usachblue!20}
\textbf{✓} & \textbf{Item} & \textbf{Estado} \\
\hline
\endfirsthead
\hline
\rowcolor{usachblue!20}
\textbf{✓} & \textbf{Item} & \textbf{Estado} \\
\hline
\endhead
$\square$ & Informe en formato PDF (15-20 páginas) & Pendiente \\
\hline
$\square$ & Repositorio GitHub público y documentado & Pendiente \\
\hline
$\square$ & Mínimo 3 mapas temáticos con elementos cartográficos & Pendiente \\
\hline
$\square$ & Mínimo 5 gráficos estadísticos & Pendiente \\
\hline
$\square$ & 1 visualización interactiva funcional & Pendiente \\
\hline
$\square$ & Datos descargados y organizados & Pendiente \\
\hline
$\square$ & Código Python ejecutable y documentado & Pendiente \\
\hline
$\square$ & Análisis exploratorio completo (EDA) & Pendiente \\
\hline
$\square$ & Marco teórico con mínimo 10 referencias & Pendiente \\
\hline
$\square$ & Cronograma detallado (Gantt) & Pendiente \\
\hline
$\square$ & Presentación slides (PDF o PPTX) & Pendiente \\
\hline
$\square$ & README.md con instrucciones de reproducción & Pendiente \\
\hline
$\square$ & requirements.txt o environment.yml & Pendiente \\
\hline
$\square$ & Evidencia de trabajo colaborativo (commits de todos los integrantes) & Pendiente \\
\hline
$\square$ & Diagrama de flujo metodológico & Pendiente \\
\hline
\end{longtable}

\section{Rúbrica de Evaluación Detallada}

\subsection{Niveles de Desempeño}

\begin{table}[H]
\centering
\small
\begin{tabularx}{\textwidth}{|l|X|X|X|X|}
\hline
\rowcolor{usachblue!20}
\textbf{Criterio} & \textbf{Excelente (7.0)} & \textbf{Bueno (6.0)} & \textbf{Suficiente (5.0)} & \textbf{Insuficiente (<4.0)} \\
\hline
\textbf{Comprensión del Problema} &
Demuestra comprensión profunda, identifica complejidades y propone soluciones innovadoras &
Comprende bien el problema y propone soluciones adecuadas &
Comprensión básica con soluciones estándar &
Comprensión limitada o incorrecta \\
\hline
\textbf{Calidad de Datos} &
Múltiples fuentes integradas, datos actualizados, metadata completa &
Buenas fuentes, datos apropiados &
Datos mínimos necesarios &
Datos insuficientes o inadecuados \\
\hline
\textbf{Análisis Técnico} &
Análisis sofisticado, métodos avanzados bien aplicados &
Buen análisis con métodos apropiados &
Análisis básico pero correcto &
Análisis superficial o con errores \\
\hline
\textbf{Visualizaciones} &
Excepcionales, claras, estéticamente superiores, interactivas &
Buena calidad, claras y bien diseñadas &
Correctas pero básicas &
Pobres o con errores \\
\hline
\textbf{Código} &
Limpio, bien documentado, modular, eficiente &
Bien estructurado y funcional &
Funciona pero poco organizado &
Desorganizado o no funciona \\
\hline
\textbf{Presentación} &
Clara, fluida, dominio del tema, responde preguntas con solidez &
Buena presentación, responde bien &
Presentación correcta &
Confusa o incompleta \\
\hline
\end{tabularx}
\caption{Rúbrica de evaluación por criterios}
\end{table}

\section{Ejemplos de Proyectos Exitosos}

\subsection{Características de Proyectos Destacados}

Para alcanzar la excelencia (nota 6.5-7.0), los proyectos anteriores han incluido:

\begin{itemize}
    \item \textbf{Integración de múltiples fuentes:} OSM + Census + Sentinel + DEM
    \item \textbf{Análisis espacial avanzado:} Hot spots, clusters, interpolación
    \item \textbf{Machine Learning básico:} Clasificación o predicción espacial
    \item \textbf{Dashboard interactivo:} Streamlit con mapas Folium
    \item \textbf{Reproducibilidad total:} Docker o ambiente virtual completo
    \item \textbf{Impacto social:} Problema real con aplicación práctica
\end{itemize}

\section{Recursos de Apoyo}

\subsection{Herramientas Recomendadas}

\begin{itemize}
    \item \textbf{Adquisición de datos:}
    \begin{itemize}
        \item OSMnx para OpenStreetMap
        \item Google Earth Engine para imágenes satelitales
        \item IDE Chile para datos oficiales
        \item Census API para datos demográficos
    \end{itemize}

    \item \textbf{Procesamiento:}
    \begin{itemize}
        \item GeoPandas para vectores
        \item Rasterio para rasters
        \item PySAL para análisis espacial
    \end{itemize}

    \item \textbf{Visualización:}
    \begin{itemize}
        \item Folium para mapas web
        \item Matplotlib/Seaborn para gráficos
        \item Plotly para gráficos interactivos
        \item Streamlit para dashboards
    \end{itemize}
\end{itemize}

\subsection{Plantillas Disponibles}

En el repositorio del curso encontrarán:
\begin{itemize}
    \item Template LaTeX para el informe
    \item Estructura base de proyecto Python
    \item Ejemplos de notebooks bien documentados
    \item Scripts de descarga de datos
\end{itemize}

\section{Fechas Importantes}

\begin{tcolorbox}[colback=warningyellow!20, colframe=warningyellow!80]
\begin{itemize}
    \item \textbf{Publicación de esta guía:} \today
    \item \textbf{Consultas y dudas:} Durante las próximas 2 semanas en clase
    \item \textbf{Presentaciones:} 7 de octubre de 2025
    \item \textbf{Entrega de informe y código:} 10 de octubre de 2025
    \item \textbf{Retroalimentación:} Una semana después de las presentaciones
\end{itemize}
\end{tcolorbox}

\section{Criterios de Penalización}

\begin{warning}
Las siguientes situaciones resultarán en penalizaciones:
\end{warning}

\begin{itemize}
    \item \textbf{Entrega tardía:} -1.0 punto por día de atraso
    \item \textbf{Falta de algún integrante a la presentación:} -0.5 puntos
    \item \textbf{Código no reproducible:} -0.5 puntos
    \item \textbf{Sin commits de algún integrante:} -0.3 puntos
    \item \textbf{Plagio o copia:} Nota 1.0 y revisión por comité de ética
\end{itemize}

\section{Preguntas Frecuentes}

\subsection{¿Qué pasa si nuestros datos aún no están completos?}
Deben mostrar avance con los datos disponibles y explicar claramente el plan para obtener los faltantes. Es mejor ser transparente sobre las limitaciones.

\subsection{¿Podemos cambiar el enfoque del proyecto?}
Sí, pero debe justificarse en el informe. Esta evaluación es precisamente para detectar si se necesitan ajustes.

\subsection{¿Qué nivel de análisis espacial se espera?}
Como mínimo: estadísticas descriptivas espaciales, mapas de densidad/distribución, y algún índice espacial (Moran's I, Getis-Ord, etc.).

\subsection{¿El código debe estar 100\% terminado?}
No, pero debe ser funcional para los análisis presentados y estar bien documentado. Se espera un 40-50\% de avance del proyecto total.

\section{Contacto y Consultas}

\begin{tcolorbox}[colback=infoblue!20, colframe=infoblue!80]
\textbf{Horarios de consulta:}
\begin{itemize}
    \item Presencial: Horario a coordinar
    \item Zoom: Previa coordinación por email
    \item Email: francisco.parra@usach.cl
\end{itemize}
\end{tcolorbox}

\vspace{1cm}

\begin{center}
\large{\textbf{¡Éxito en su primera evaluación!}}
\end{center}

\end{document}