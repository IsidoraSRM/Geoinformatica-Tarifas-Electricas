% ====================================================================
% PLANTILLA INFORME PRIMERA ENTREGA - GEOINFORMÁTICA USACH
% Autor: Prof. Francisco Parra O.
% ====================================================================

\documentclass[11pt,letterpaper]{article}

% PAQUETES ESENCIALES
\usepackage[utf8]{inputenc}
\usepackage[spanish,es-tabla]{babel}
\usepackage[left=2.5cm,right=2.5cm,top=2.5cm,bottom=2.5cm]{geometry}
\usepackage{graphicx}
\usepackage{float}
\usepackage{xcolor}
\usepackage{hyperref}
\usepackage{amsmath,amssymb}
\usepackage{booktabs}
\usepackage{caption}
\usepackage{subcaption}
\usepackage{longtable}
\usepackage{array}
\usepackage{multirow}
\usepackage{listings}
\usepackage{fancyhdr}
\usepackage{lastpage}
\usepackage[backend=biber,style=apa]{biblatex}

% Configuración de bibliografía
\addbibresource{referencias.bib} % Crear este archivo para las referencias

% Colores institucionales
\definecolor{usachblue}{RGB}{0,46,93}
\definecolor{usachred}{RGB}{200,16,46}
\definecolor{codegreen}{rgb}{0,0.6,0}
\definecolor{codegray}{rgb}{0.5,0.5,0.5}
\definecolor{codepurple}{rgb}{0.58,0,0.82}
\definecolor{backcolour}{rgb}{0.95,0.95,0.92}

% Configuración de código
\lstdefinestyle{mystyle}{
    backgroundcolor=\color{backcolour},
    commentstyle=\color{codegreen},
    keywordstyle=\color{magenta},
    numberstyle=\tiny\color{codegray},
    stringstyle=\color{codepurple},
    basicstyle=\ttfamily\footnotesize,
    breakatwhitespace=false,
    breaklines=true,
    captionpos=b,
    keepspaces=true,
    numbers=left,
    numbersep=5pt,
    showspaces=false,
    showstringspaces=false,
    showtabs=false,
    tabsize=2,
    language=Python
}
\lstset{style=mystyle}

% Configuración de hyperref
\hypersetup{
    colorlinks=true,
    linkcolor=usachblue,
    filecolor=magenta,
    urlcolor=usachblue,
    citecolor=usachblue
}

% Encabezado y pie de página
\pagestyle{fancy}
\fancyhf{}
\lhead{\textcolor{usachblue}{Geoinformática - Primera Entrega}}
\rhead{\textcolor{usachblue}{USACH 2025}}
\lfoot{\small\textit{[Nombre del grupo]}}
\cfoot{\thepage\ de \pageref{LastPage}}
\rfoot{\today}

% ====================================================================
% PORTADA - MODIFICAR CON SUS DATOS
% ====================================================================

\begin{document}

\begin{titlepage}
    \centering
    \vspace*{1cm}

    \Large{\textbf{UNIVERSIDAD DE SANTIAGO DE CHILE}}\\
    \large{Facultad de Ingeniería}\\
    \large{Departamento de Ingeniería Informática}\\

    \vspace{3cm}

    \rule{\textwidth}{0.5mm}\\
    \vspace{0.5cm}

    {\LARGE\textbf{[TÍTULO DE SU PROYECTO]}}\\
    \vspace{0.3cm}
    {\Large\textit{[Subtítulo descriptivo opcional]}}\\

    \vspace{0.5cm}
    \rule{\textwidth}{0.5mm}

    \vspace{2cm}

    {\Large\textbf{Primera Evaluación}}\\
    \vspace{0.5cm}
    {\large Informe de Avance}

    \vspace{3cm}

    \begin{minipage}{0.8\textwidth}
        \begin{flushleft}
            \large
            \textbf{Integrantes del Grupo:}\\
            [Nombre Completo 1] - [RUT]\\
            [Nombre Completo 2] - [RUT]\\
            [Nombre Completo 3] - [RUT]\\
            [Nombre Completo 4] - [RUT]\\
            [Nombre Completo 5] - [RUT]\\
            [Agregar más si es necesario]\\
            \vspace{0.5cm}
            \textbf{Profesor:}\\
            Francisco Parra O.\\
            \vspace{0.5cm}
            \textbf{Ayudantes:}\\
            [Si aplica]
        \end{flushleft}
    \end{minipage}

    \vfill

    {\large \today}
\end{titlepage}

% ====================================================================
% RESUMEN EJECUTIVO
% ====================================================================

\newpage
\section*{Resumen Ejecutivo}
\addcontentsline{toc}{section}{Resumen Ejecutivo}

% [Máximo 1 página]
% Incluir: problema abordado, área de estudio, metodología principal,
% resultados preliminares clave, y próximos pasos.

Este proyecto aborda [describir el problema]... El área de estudio comprende [ubicación]...
La metodología empleada incluye [métodos principales]... Los resultados preliminares
muestran [hallazgos clave]... Los próximos pasos contemplan [siguientes fases]...

% ====================================================================
% ÍNDICE
% ====================================================================

\newpage
\tableofcontents
\newpage

% ====================================================================
% 1. INTRODUCCIÓN
% ====================================================================

\section{Introducción}

\subsection{Contexto y Motivación}

% Explicar el contexto del problema, por qué es importante,
% y qué motivó la selección de este proyecto

\subsection{Objetivos}

\subsubsection{Objetivo General}

% Un objetivo general claro y alcanzable

\subsubsection{Objetivos Específicos}

\begin{enumerate}
    \item Objetivo específico 1...
    \item Objetivo específico 2...
    \item Objetivo específico 3...
    \item Objetivo específico 4...
\end{enumerate}

\subsection{Preguntas de Investigación}

\begin{itemize}
    \item ¿Pregunta 1...?
    \item ¿Pregunta 2...?
    \item ¿Pregunta 3...?
\end{itemize}

\subsection{Alcance y Limitaciones}

% Definir claramente qué incluye y qué no incluye el proyecto

% ====================================================================
% 2. MARCO TEÓRICO
% ====================================================================

\section{Marco Teórico}

\subsection{Antecedentes Conceptuales}

% Conceptos clave de geoinformática relevantes al proyecto

\subsection{Estado del Arte}

% Revisión de literatura, proyectos similares, metodologías aplicadas

\subsection{Marco Metodológico de Referencia}

% Métodos y técnicas que se usarán basados en la literatura

% Ejemplo de cita: \parencite{autor2023}

% ====================================================================
% 3. ÁREA DE ESTUDIO
% ====================================================================

\section{Área de Estudio}

\subsection{Ubicación Geográfica}

% Descripción precisa con coordenadas

\begin{figure}[H]
    \centering
    % \includegraphics[width=0.8\textwidth]{figuras/mapa_ubicacion.png}
    \caption{Mapa de ubicación del área de estudio}
    \label{fig:ubicacion}
\end{figure}

\subsection{Características del Territorio}

\subsubsection{Características Físicas}

% Topografía, clima, hidrografía, etc.

\subsubsection{Características Socioeconómicas}

% Población, actividades económicas, etc.

\subsection{Justificación de la Selección}

% Por qué esta área es apropiada para el estudio

% ====================================================================
% 4. DATOS Y METODOLOGÍA
% ====================================================================

\section{Datos y Metodología}

\subsection{Fuentes de Datos}

\begin{table}[H]
\centering
\caption{Resumen de fuentes de datos utilizadas}
\label{tab:datos}
\begin{tabular}{@{}llllll@{}}
\toprule
\textbf{Dato} & \textbf{Fuente} & \textbf{Formato} & \textbf{Resolución} & \textbf{Fecha} & \textbf{Acceso} \\
\midrule
OSM & OpenStreetMap & .osm/shapefile & Vectorial & 2024 & API/OSMnx \\
DEM & NASA SRTM & .tif & 30m & 2023 & USGS \\
Censo & INE Chile & .csv & Manzana & 2017 & Portal INE \\
Sentinel-2 & Copernicus & .tif & 10m & 2024 & GEE \\
% Agregar más filas según necesidad
\bottomrule
\end{tabular}
\end{table}

\subsection{Preprocesamiento de Datos}

\subsubsection{Limpieza y Validación}

% Describir procesos de limpieza

\subsubsection{Transformaciones y Proyecciones}

% Sistemas de coordenadas, reproyecciones

\subsection{Metodología de Análisis}

\subsubsection{Diagrama de Flujo Metodológico}

\begin{figure}[H]
    \centering
    % \includegraphics[width=\textwidth]{figuras/diagrama_flujo.png}
    \caption{Diagrama de flujo metodológico del proyecto}
    \label{fig:flujo}
\end{figure}

\subsubsection{Análisis Espacial}

% Describir técnicas de análisis espacial a utilizar

\subsubsection{Análisis Estadístico}

% Métodos estadísticos y geoestadísticos

% ====================================================================
% 5. RESULTADOS PRELIMINARES
% ====================================================================

\section{Resultados Preliminares}

\subsection{Análisis Exploratorio de Datos}

\subsubsection{Estadísticas Descriptivas}

\begin{table}[H]
\centering
\caption{Estadísticas descriptivas de variables principales}
\begin{tabular}{@{}lrrrrr@{}}
\toprule
\textbf{Variable} & \textbf{Media} & \textbf{Mediana} & \textbf{Desv.Est} & \textbf{Min} & \textbf{Max} \\
\midrule
Variable 1 & 0.00 & 0.00 & 0.00 & 0.00 & 0.00 \\
Variable 2 & 0.00 & 0.00 & 0.00 & 0.00 & 0.00 \\
\bottomrule
\end{tabular}
\end{table}

\subsubsection{Distribuciones Espaciales}

\begin{figure}[H]
    \centering
    \begin{subfigure}[b]{0.45\textwidth}
        % \includegraphics[width=\textwidth]{figuras/mapa1.png}
        \caption{Distribución de variable 1}
    \end{subfigure}
    \hfill
    \begin{subfigure}[b]{0.45\textwidth}
        % \includegraphics[width=\textwidth]{figuras/mapa2.png}
        \caption{Distribución de variable 2}
    \end{subfigure}
    \caption{Mapas de distribución espacial}
    \label{fig:distribucion}
\end{figure}

\subsection{Primeros Análisis}

\subsubsection{Autocorrelación Espacial}

% Índice de Moran, LISA, etc.

\subsubsection{Identificación de Patrones}

% Hot spots, clusters, tendencias

\subsection{Visualizaciones Interactivas}

% Descripción y link a visualizaciones web si existen

% ====================================================================
% 6. IMPLEMENTACIÓN TÉCNICA
% ====================================================================

\section{Implementación Técnica}

\subsection{Arquitectura del Sistema}

% Describir la estructura del código y sistema

\subsection{Herramientas y Librerías}

\begin{itemize}
    \item \textbf{Lenguaje:} Python 3.10+
    \item \textbf{Geoespacial:} GeoPandas, Shapely, Rasterio, OSMnx
    \item \textbf{Análisis:} NumPy, Pandas, Scikit-learn, PySAL
    \item \textbf{Visualización:} Matplotlib, Folium, Plotly
    \item \textbf{Base de datos:} PostGIS (si aplica)
\end{itemize}

\subsection{Código Ejemplo}

\begin{lstlisting}[language=Python, caption=Ejemplo de código para análisis espacial]
import geopandas as gpd
import matplotlib.pyplot as plt
from pysal.explore import esda

# Cargar datos
gdf = gpd.read_file('data/area_estudio.shp')

# Calcular autocorrelación espacial
moran = esda.Moran(gdf['variable'], weights)
print(f"Moran's I: {moran.I:.4f}")
print(f"P-value: {moran.p_norm:.4f}")
\end{lstlisting}

\subsection{Reproducibilidad}

% Link al repositorio GitHub
El código completo está disponible en: \url{https://github.com/usuario/proyecto}

% ====================================================================
% 7. DISCUSIÓN
% ====================================================================

\section{Discusión}

\subsection{Interpretación de Resultados Preliminares}

% Qué significan los resultados obtenidos hasta ahora

\subsection{Desafíos Encontrados}

\begin{enumerate}
    \item \textbf{Desafío 1:} Descripción y cómo se abordó/abordará
    \item \textbf{Desafío 2:} Descripción y cómo se abordó/abordará
    \item \textbf{Desafío 3:} Descripción y cómo se abordó/abordará
\end{enumerate}

\subsection{Ajustes al Plan Original}

% Si hubo cambios respecto a la propuesta inicial

% ====================================================================
% 8. CRONOGRAMA
% ====================================================================

\section{Cronograma de Actividades}

\subsection{Actividades Completadas}

\begin{table}[H]
\centering
\caption{Actividades completadas hasta la fecha}
\begin{tabular}{@{}llll@{}}
\toprule
\textbf{Actividad} & \textbf{Responsable} & \textbf{Inicio} & \textbf{Fin} \\
\midrule
Selección del proyecto & Todo el grupo & Semana 1 & Semana 1 \\
Revisión bibliográfica & [Nombre] & Semana 2 & Semana 3 \\
Descarga de datos & [Nombre] & Semana 3 & Semana 4 \\
Análisis exploratorio & Todo el grupo & Semana 4 & Semana 4 \\
\bottomrule
\end{tabular}
\end{table}

\subsection{Actividades Pendientes}

\begin{table}[H]
\centering
\caption{Cronograma de actividades restantes}
\begin{tabular}{@{}llll@{}}
\toprule
\textbf{Actividad} & \textbf{Responsable} & \textbf{Inicio} & \textbf{Duración} \\
\midrule
Análisis espacial avanzado & [Nombre] & Semana 5 & 2 semanas \\
Modelamiento & [Nombre] & Semana 7 & 2 semanas \\
Desarrollo dashboard & Todo el grupo & Semana 9 & 2 semanas \\
Documentación final & Todo el grupo & Semana 11 & 1 semana \\
Preparación presentación final & Todo el grupo & Semana 12 & 1 semana \\
\bottomrule
\end{tabular}
\end{table}

% También incluir diagrama de Gantt si es posible

% ====================================================================
% 9. CONCLUSIONES PRELIMINARES
% ====================================================================

\section{Conclusiones Preliminares}

\subsection{Logros Alcanzados}

\begin{itemize}
    \item Logro 1...
    \item Logro 2...
    \item Logro 3...
\end{itemize}

\subsection{Factibilidad del Proyecto}

% Evaluación de si el proyecto es realizable en el tiempo restante

\subsection{Próximos Pasos Críticos}

\begin{enumerate}
    \item Paso crítico 1...
    \item Paso crítico 2...
    \item Paso crítico 3...
\end{enumerate}

% ====================================================================
% REFERENCIAS
% ====================================================================

\newpage
\printbibliography[heading=bibintoc,title={Referencias}]

% ====================================================================
% ANEXOS
% ====================================================================

\newpage
\appendix
\section{Anexos}

\subsection{Anexo A: Código Adicional}

% Código extenso que no cabe en el cuerpo principal

\subsection{Anexo B: Tablas Complementarias}

% Tablas de datos adicionales

\subsection{Anexo C: Mapas Adicionales}

% Más visualizaciones si es necesario

\subsection{Anexo D: Enlaces y Recursos}

\begin{itemize}
    \item \textbf{Repositorio GitHub:} \url{https://github.com/usuario/proyecto}
    \item \textbf{Dashboard interactivo:} \url{http://link-al-dashboard}
    \item \textbf{Datos originales:} \url{http://link-a-datos}
\end{itemize}

\end{document}